\documentclass[10pt,letterpaper]{article}
\usepackage{graphicx}
\usepackage{url}
\usepackage{color}
\usepackage{enumerate}
\usepackage{fancyhdr}
\usepackage[resetlabels]{multibib}
\usepackage{tabularx}
\usepackage{geometry} % to change the page dimensions
\geometry{a4paper}
\geometry{margin=1in} 



% Turn off page numbering
\pagestyle{fancy}
\fancyhf{}
\cfoot{\datestyle \thepage}
\rfoot{\datestyle Resume last updated:  \today}
\renewcommand{\headrulewidth}{0.0pt} % remove bar

% Set parameters and content.
\newcommand{\namestyle}{\Huge \scshape}
\newcommand{\deptstyle}{\footnotesize \rmfamily \scshape}
\newcommand{\addressstyle}{\footnotesize \rmfamily \upshape}
\newcommand{\datestyle}{ \footnotesize \rmfamily \upshape}

% Remove list labels.
\renewcommand{\labelitemi}{}
\renewcommand{\labelitemii}{}




% Define bibliographies.
\newcites{j,c,a}{Journal Publications,Conference Publications, Publications}



\usepackage[none]{hyphenat}

% Define bibliographies.
%\newcites{j,c}{Journal Publications,Conference Publications}

%\setlength\parindent{0pt}
\begin{document}
\iffalse
% Print title.
\begin{center}
\namestyle Ariel S. Anders \\[0.3em]
\deptstyle Department of Electrical Engineering and Computer Science, Massachusetts Institute of Technology \\[0.2em]
\addressstyle Address: 32 Vassar Street 32-G418 Cambridge, MA 02139\\
aanders@csail.mit.edu \ $\cdot$ www.people.csail.mit.edu/aanders
\end{center}


Dear awesome Robotics Company, \\
\\\par
I am an Electrical Engineering Computer Science doctorate student at the Massachusetts Institute of Technology seeking a robotics internship.  I am interested in pursuing areas that center around my research focus of robotic planning and manipulation. \par 
At MIT, I have investigated different facets of robotic planning and manipulation. My master’s thesis investigated applying reinforcement learning to robotic grasping.  More recently I collaborated on a multi robot planning project that was applied to several domains, including a beer delivery one.  This project, titled ``Beer Bots'' won second place at the inaugural CSAIL Research Highlights. Additionally the paper we presented for this project was a best paper finalist at RSS 2015.   \par 

My current focus is improving  humanoid robots manipulation capability.  I'm taking inspiration from large manufacturing  systems, such as the vibratory bowl feeder, which uses orientating devices to force items into a specific range of configurations.  This idea, applied in a robot domain is fundamentally different from current approaches and could yield more predictable results. Ultimately, my goal for this research is for the robot to solve long sequential manipulation problems for useful tasks like assembly or household cleaning.

Sincerely,\\ \par
Ariel Anders

\pagebreak
\fi
% Print title.
\begin{center}
\namestyle Ariel S. Anders \\[0.3em]
\deptstyle Ph.D. Candidate  at  Massachusetts Institute of Technology \\Department of Electrical Engineering and Computer Science\\[0.2em]
\addressstyle Address: 32 Vassar Street 32-G418 Cambridge, MA 02139\\
    aanders@csail.mit.edu \ $\cdot$ www.people.csail.mit.edu/aanders
\end{center}

\section*{Education}
 \subparagraph{Massachusetts Institute of Technology $\cdot$ Cambridge, MA}

\begin{itemize}
    \item {\bf Ph.D., Electrical Engineering and Computer Science $\cdot$ Fall 2014 - present}
 	\begin{itemize}
        \item Advisors: Prof. Leslie Kaelbling and Prof. Tom\'as Lozano P\'erez.
        \item Passed the Technical and Research Qualifying Examinations.
        \item {\em Completed minor: Mechanical Engineering/Aeronautics and Astronautics\\
                (courses in controls and autonomous vehicles)}
	\end{itemize}
\end{itemize}

\begin{itemize}
 	\item {\bf S.M., Electrical Engineering and Computer Science$\cdot$ Fall 2012 - Spring 2014}
    \begin{itemize}
	    \item Advisors: Prof. Leslie Kaelbling and Prof. Tom\'as Lozano P\'erez.
	    \item Masters Thesis: ``Learning a Strategy for Whole Arm Grasping''
	\end{itemize}
\end{itemize}

\begin{itemize}
    \item {\bf GPA:  4.9/5.0 }
	\item {\bf Relevant Coursework}
    \item
    \begin{tabularx}{\textwidth}{l l}
      6.867: Machine Learning 
      & \hfill 6.831: User Interface Design and Implementation\\
      6.375: Design of Complex Digital Systems
      & \hfill 6.852: Distributed Algorithms\\
      16.31: Feedback and Control Systems
      & \hfill 2.166 Autonomous Vehicles\\
      %6.811 Principles and Practices of Assistive Technology
     \end{tabularx}
\end{itemize}


\subparagraph{University of California, Santa Cruz $\cdot$ Santa Cruz, CA }
\begin{itemize}
    \item {\bf B. S., Computer Engineering$\cdot$ Fall 2008 - Spring 2012}
    \begin{itemize}
		\item {\em University of California Regent Scholar} 
		\item University Honors: {\em Summa Cum Laude}, Department Honors: Highest Honors in the major

        \item {\bf Senior Design Capstone Project}\\
        Team project to improve the performance of arithmetic functions for Oracle numbers within the Oracle Database; this software development project was completely done in C on x86 and ARM processors using code profilers to find performance bottle necks and applying vectorized hardware instructions (SSE) and different number representations to achieve speedup. 
    \end{itemize}
\end{itemize}

\begin{itemize}
    \item {\bf GPA: 3.96/4.00 }
    \item {\bf Relevant Coursework}
    \item
    \begin{tabularx}{\textwidth}{l l}
        CMPE 215: Models of Robotic Manipulation 
        & \hfill CMPE 118: Introduction to Mechatronics \\
        CMPE 121: Microprocessor System Design 
        & \hfill CMPE 110: Computer Architecture \\
        CMPE 100: Logic Design 
        & \hfill EE 101: Electornic Circuits \\
         AMS 114: Introduction to Dynamical Systems 
        & \hfill EE 103: Signals and Systems \\
        EE 154: Feedback Control Systems 
        & \hfill CMPS 101: Algorithms and Abstract Data Types\\
    \end{tabularx}
\end{itemize}



\section*{Professional Experience}
\subparagraph{Lead Technology Developer, LEAC} MIT $\cdot$ Cambridge, MA $\cdot$ January 2017- Present
\begin{itemize}
    \item The Lab Energy Assessment Center (LEAC) provides low cost and minimally invasive tools to detect and analyze energy inefficiencies. I develop scalable software for wireless power monitoring for tools. All software is available open source at \url{https://github.com/leac-mit}.
\item See \url{https://leac.mit.edu} for more information
\end{itemize}

\subparagraph{Software Engineering Intern, Intel}
Santa Clara, CA $\cdot$ Summer 2014
\begin{itemize}
\item Responsibilities included designing, writing, testing, and documenting design automation software that uses machine learning techniques to determine proper and efficient simulation points. These simulation points are used during architecture analysis of future Intel Architecture based products and platforms.
\end{itemize}


\section*{Research Experience}
\subparagraph{Graduate Research Assistant, CSAIL}
MIT$\cdot$ Cambridge, MA $\cdot$ Summer 2012 - Present
\begin{itemize}
	\item Research focus: planning, reinforcement learning and robotic manipulation
%	\item Used machine learning and Artificial Intelligence techniques to implement whole-limb manipulation for grasping large objects with the PR2 robot.
 \end{itemize}
 \subparagraph{Undergraduate Research, Bionics Lab}
UCSC $\cdot$ Santa Cruz, CA $\cdot$ Summer 2010 - Spring 2012
\begin{itemize}
	\item Advisor: Jacob Rosen 
     	\item Research focus: CAD/CAM applications in dentistry, autonomous control with mechanical systems, and UI development for robotic programs.
     	\item  Developed a workflow to execute dental crowning and implant placement procedures on static dental models that I verified experimentally. Worked on a system to implement dynamic dental procedures. 
\end{itemize}



\section*{Teaching Experience}
\subparagraph{6.141 Robotics: Science and Systems I Teaching Assistant}
MIT$\cdot$ Spring 2017
\begin{itemize}
\item Course Description: Presents concepts, principles, and algorithms for sensing and computation related to the physical world. Topics include motion planning, geometric reasoning, kinematics and dynamics, state estimation, tracking, map building, manipulation, human-robot interaction, fault diagnosis, and embedded system development. Students specify and design a small-scale yet complex robot capable of real-time interaction with the natural world. 
\end{itemize}
\subparagraph{Beaverworks Summer Institute Lead Associate Instructor}
 $\cdot$ Summer 2016
\begin{itemize}
\item The MIT Beaver Works Summer Institute is a 4-week residential STEM-based program for talented rising high school seniors (entering the 12th grade). This years exciting project is the MIT Mini Grand Prix Challenge, a hands-on, intensive 4-week program that will focus on demonstrating fast, autonomous navigation of small racecars in a complex environment. As lead associate instructor I lead a team of 6 associate instructors during the lab portions of the course.  Additionally, I assisted in creating lab curriculum throughout the course and lead the material creation for the computer vision session of the course. See racecar.mit.edu for more details.
\end{itemize}
\subparagraph{6.141 Robotics: Science and Systems I Teaching Assistant}
MIT$\cdot$ Spring 2016
\begin{itemize}
\item Course Description: Presents concepts, principles, and algorithms for sensing and computation related to the physical world. Topics include motion planning, geometric reasoning, kinematics and dynamics, state estimation, tracking, map building, manipulation, human-robot interaction, fault diagnosis, and embedded system development. Students specify and design a small-scale yet complex robot capable of real-time interaction with the natural world. 
\item {\em Course TA for the first time this course was offered with the new RACECAR platform}
\end{itemize}
\subparagraph{6.01 Intro to EECS Teaching Assistant}
MIT$\cdot$ Spring 2015
\begin{itemize}
\item Course Description: An integrated introduction to electrical engineering and computer science, taught using substantial laboratory experiments with mobile robots. Key issues in the design of engineered artifacts operating in the natural world: measuring and modeling system behaviors; assessing errors in sensors and effectors; specifying tasks; designing solutions based on analytical and computational models; planning, executing, and evaluating experimental tests of performance; refining models and designs. Issues addressed in the context of computer programs, control systems, probabilistic inference problems, circuits and transducers, which all play important roles in achieving robust operation of a large variety of engineered systems.
\end{itemize}
\subparagraph{Math 3 Precalculus  Teaching Assistant}
UCSC $\cdot$ Spring 2010 \& 2011, Fall 2011, Winter 2012
\begin{itemize}
\item Course Description:Inverse functions and graphs; exponential and logarithmic functions, their graphs, and use in mathematical models of the real world; rates of change; trigonometry, trigonometric functions, and their graphs; and geometric series. 
\end{itemize}

\subparagraph{Math 2  College Algebra for Calculus Teaching Assistant}
UCSC $\cdot$ Fall 2009, Winter 2010
\begin{itemize}
\item Course Description: Operations on real numbers, complex numbers, polynomials, and rational expressions; exponents and radicals; solving linear and quadratic equations and inequalities; functions, algebra of functions, graphs; conic sections; mathematical models; sequences and series.
\end{itemize}


\subparagraph{Math 2 Stretch Teaching Assistant}
UCSC $\cdot$ Fall 2010 - Winter 2011
\begin{itemize}
\item Course Description: This two-credit, stretch course offers students two quarters to master material covered in course 2: operations on real numbers, complex numbers, polynomials, and rational expressions; exponents and radicals; solving linear and quadratic equations and inequalities; functions, algebra of functions, graphs; conic sections; mathematical models; sequences and series. After successful completion of this course in the first quarter, students enroll in course 2 the following quarter to complete the sequence and earn an additional 5 credits.
\item {\em Course TA for the first time this course was offered}
\end{itemize}

\subparagraph{Academic Excellence Co-leader for Math 19A}
UCSC $\cdot$ Winter 2009
\begin{itemize}
\item Course Title: Calculus for Science, Engineering, and Mathematics. 
\item Course Description: The limit of a function, calculating limits, continuity, tangents, velocities, and other instantaneous rates of change. Derivatives, the chain rule, implicit differentiation, higher derivatives. Exponential functions, inverse functions, and their derivatives. The mean value theorem, monotonic functions, concavity, and points of inflection. Applied maximum and minimum problems. Inverse functions and graphs; exponential and logarithmic functions, their graphs, and use in mathematical models of the real world; rates of change; trigonometry, trigonometric functions, and their graphs; and geometric series. 
\item See \url{http://ace.ucsc.edu} for more details.
\end{itemize}

\subparagraph{Academic Excellence Co-leader for Math 3}
UCSC $\cdot$ Fall 2008
\begin{itemize}
\item Course Title: Precalculus
\item Course Description:Inverse functions and graphs; exponential and logarithmic functions, their graphs, and use in mathematical models of the real world; rates of change; trigonometry, trigonometric functions, and their graphs; and geometric series. 
\item See \url{http://ace.ucsc.edu} for more details.
\end{itemize}

\subparagraph{Girls in Engineering Robotics Instructor}
UCSC $\cdot$ Summer 2009
\begin{itemize}
\item Summercamp for middle school girls to learn how to program Lego Mindstorm Robots. Prepared course material on programming the robots and an introduction to Processing.
\end{itemize}

\bibliographystylea{abbrv}
\nocitea{acmp2017}
\nocitea{mems17}
\nocitea{isec17}
\nocitea{ijrr17}
\nocitea{apl16}
\nocitea{eaai17}
\nocitea{rss15}
\nocitea{swag}
\nocitea{dental}
\bibliographya{ari}

\iffalse
    \bibliographystylej{IEEEtran}
    \bibliographystylec{IEEEtran}

    \nocitej{ijrr17}
    \nocitec{rss15}
    \nocitec{eaai17}
    \nocitec{dental}
    \nocitej{apl16}
    \nocitec{mems17}
    \nocitec{isec17}

    \bibliographyj{ari}
    \bibliographyc{ari}
\fi


\section*{Research Fellowships}
\subparagraph{Graduate Research Fellowships}
MIT$\cdot$ Cambridge, MA 
\begin{itemize}
\item GEM Ph.D. Engineering Fellowship Sponsored by Intel, Summer 2014
\item  Edwin S. Webster Graduate Fellowship in Electrical Engineering, Spring 2013
\item Lemelson Minority Graduate Fellowship, Fall 2012
\end{itemize}
\subparagraph{Undergraduate Research Fellowships}
UCSC $\cdot$ Santa Cruz, CA
\begin{itemize}
\item Minority Access to Research Careers, Summer 2010- Spring 2012
\item  Summer Undergraduate Research Fellowship in Information Technology, Summer 2010.  
\end{itemize}


\section*{Honors and Awards}
\subparagraph{Grant Recipient}
\begin{itemize}
\item 2017 Earth Day Mini Grant \\
Award for developing new fume hood technologies, in partnership with LEAC at MIT.
\item 2017 MIT Green Labs Innovation Award\\
\$5000 Award received in collaboration with Daniel Preston and the Device Research Lab for developing most innovative technology to improve sustainability efforts at campus at MIT. 
\item 2016 MIT EHS Green Labs Award \\
    Received \$1000 in seed funding to create green lab technology. Award received in collaboration with Daniel Preston and the Device Research Lab. 
\item MindHandHeart Innovation Fund Grant Recipient, Fall 2015\\
``Removing SAD from Winter", {\em Planning for public artificial lightbox locations on campus for people with Seasonal Affective Disorder }
\item University Center of Exemplary Mentoring at MIT Scholar, Innaugural class of 2015.
\end{itemize}

\subparagraph{Academic Honors and Scholarships}
\begin{itemize}
\item MIT Graduate Women of Excellence, class of 2017 Honorees
\item University of California Regent Scholarship, Fall 2010-Spring 2012 
\item  Google Travel Scholarship for NSBE,  Winter 2012 
\item Mantey Undergraduate Leadership Award,  Spring 2011 
 \item ARGV Scholarship, Spring 2010 
\item  Science Learning Community GPA Award, Spring 2009 
\item Travel Scholarship Recipient for SACNAS, Fall 2008 
\end{itemize}

\subparagraph{Research Presentation Awards}
\begin{itemize}
\item 1st Place Poster Presentation, NSBE Technical Research Exhibit, NSBE Annual conference 2012
\item Special Merit in Research Award,  2011 CAMP Symposium 
\item  National Poster Presentation Award, 2010 ABRCMS Annual Conference
\item Best Poster Design, UCSC, 2010 Undergraduate Research Symposium
\end{itemize}

\subparagraph{Tech Competitions}
\begin{itemize}
\item {\em Boop}, 4th place Assitive Technology Hackathon, Spring 2016
\item {\em lingui-sense}, 1st place at Make Cool Shmit, Spring 2016
\item {\em Haptic++}, 2nd place at Meet++ Hackathon, Spring 2016
\item {\em Beer Bots,} 2nd place at CSAIL Research Highlights, Spring 2015 
\end{itemize}




\section*{Service and Leadership Activities}
\subparagraph{Student Organization Activities}
\begin{itemize}
	\item RoboCon 2016 Committee chairperson 
	\item CSAIL Student Committee, President, AI Representative, Publicity Czar  
	\item MIT Concert Band,  Vice President and Tour Manager 
	\item Ashdown House Officer,  Coffee Hour Officer, Events Committee Officer 
	\item MIT EECS Prospective Students volunteer (Robotics RAISINS Organizer)
	\item MIT Rowing Club,  Avid Rower 
	\item MIT GSC Activities Committee, Committee Member 
	\item CSAIL Student Workshop 2012 \& 2013, Committee Chairperson  
	\item UCSC National Society of Black Engineers,   President (2 years), Peer Adviser 
	\item UCSC Society of Women Engineers , Treasurer, Undergrad Hardware Lab Representative 
	\item UCSC Tau Beta Pi,  Corresponding Secretary, Exec. Board Member  
\end{itemize}
\subparagraph{Volunteer Activities}
\begin{itemize}     
  \item LIS Robot Tour guide 2012-present
	\item UCSC NSBE Math Boot Camp Tutor  2012     
	\item UCSC Google Student Ambassador  2009-2011  
	\item Mathematics Engineering and Science Achievement Judge   2009-2011  
	\item Expanding Your Horizons conference workshop liaison  2009-2010
\end{itemize}


\section*{Computer Skills}

\begin{itemize}
	\item \textbf{Languages} \ Python, C, C++, Java, Matlab, Perl, HTML, shell script, Javascript
	\item \textbf{Software} \ Robot Operating System, SolidWorks, Simics, BlueSpec, Wincaps
	\item \textbf{OSs} \ Unix, Linux,  Mac, Windows
	\item \textbf{Robot platforms} \ Denso VM-series, PR2, TurtleBot, DuckieBot, 6.141 Racecar
	\item \textbf{Embedded Systems} Arduino, Raspberry Pi, Jetson TX1, Pandaboard, Microchip PIC 32, Virtex5 FPGA, and  68HC11E1 Microcontroller 
\end{itemize}


\section*{Personal Projects Portfolio}
\begin{itemize}
\item RACECAR Web designer: \url{http://racecar.mit.edu}
\item Boop Light Detector, iOS phone app (4+ stars, over 2100 downloads)
\item Green Net - reducing energy consumption at MIT
\item RoboCon 2016 Web Designer: http://robocon.mit.edu/
\item Ariel Anders personal webpage: http://people.csail.mit.edu/aanders/
\item More projects listed at http://people.csail.mit.edu/aanders/projects.html:
\item Open source software: https://github.com/arii
\end{itemize}

\section*{Miscellaneous}

\begin{itemize}
\item Born in California, USA. United States citizen.
\end{itemize}

\end{document}
